\documentclass{article}
\usepackage{amsfonts}
\usepackage{amsmath}
\usepackage{mathrsfs}
\usepackage{amsthm}
% \smartqed  % flush right qed marks, e.g. at end of proof
% \usepackage{graphicx}
%\newtheorem{definition}[definition]{Definition}

% \pagenumbering{gobble}
\newtheorem{theorem}{Theorem}
\newtheorem{definition}[theorem]{Definition}
\newtheorem{remark}[theorem]{Remark}
\newtheorem{lemma}[theorem]{Lemma}
\newtheorem{question}[theorem]{Question}
\newtheorem{example}[theorem]{Example}
\newtheorem{proposition}[theorem]{Proposition}
\newtheorem{corollary}[theorem]{Corollary}
% \newtheorem{definition}[theorem]{Definition}
\newtheorem{principle}[theorem]{Principle}
\newtheorem{conjecture}[theorem]{Conjecture}

\begin{document}

\title{ Crowd-wisdom of universal reinforcement learning agents }
% \titlerunning{Agent mixtures and genericness}
\author{Samuel Allen Alexander \& Len Du \& Marcus Hutter}

% \institute{The U.S.\ Securities and Exchange Commission
% \email{samuelallenalexander@gmail.com}
% \url{https://philpeople.org/profiles/samuel-alexander/publications}}

\maketitle

\begin{abstract}
The wisdom, or intelligence, of a crowd, or a collection of intelligent agents, has long been an attractive topic with an ancient philosophic root and profound implications in a wide variety of disciplines, such as economics, political science, psychology, cognitive science, behavioural science, and many branches of social sciences.

While from a very different group of scientists, we deem the current artificial intelligence research,
signified the utmost by Deep Reinforcement Learning,
to carry previously underexplored potential for addressing such ``human'' problems beneath its theoretic branch,
with its deep roots in highly rigorous mathematics and formal logic,
as abstract models of universal intelligence agents would apply to artificial and natural agents alike.

We introduce a weighted mixture operation on
reinforcement learning agents. The mixture of several weighted agents is
an agent with the
following property: the expected total reward the mixture agent
gets in any environment is the corresponding weighted average
of the expected total rewards the original agents get in that
environment. We use mixture agents to formalize and
strengthen an informal result of Alexander and Hutter. We also use mixture
agents to prove additional results, including a surprising result
which we call the genericness of non-deterministic intelligence. Loosely:
any particular non-deterministic action an agent takes
can have its probabilities modified without making the agent less
intelligent.
\end{abstract}

%\bibliographystyle{alpha}
%\bibliography{main}

\end{document}
